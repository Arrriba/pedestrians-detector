%latex model.tex
%bibtex model
%latex model.tex
%latex model.tex
%pdflatex model.tex

%se poate lucra si online (de ex www.overleaf.com)


\documentclass[runningheads,a4paper,11pt]{report}

\usepackage{algorithmic}
\usepackage{algorithm} 
\usepackage{array}
\usepackage{amsmath}
\usepackage{amsfonts}
\usepackage{amssymb}
\usepackage{amsthm}
\usepackage{caption}
\usepackage{comment} 
\usepackage{epsfig} 
\usepackage{fancyhdr}
\usepackage[T1]{fontenc}
\usepackage{geometry} 
\usepackage{graphicx}
\usepackage[colorlinks]{hyperref} 
\usepackage[latin1]{inputenc}
\usepackage{multicol}
\usepackage{multirow} 
\usepackage{rotating}
\usepackage{setspace}
\usepackage{subfigure}
\usepackage{url}
\usepackage{verbatim}
\usepackage{xcolor}
\usepackage[outdir=./]{epstopdf}

\geometry{a4paper,top=3cm,left=2cm,right=2cm,bottom=3cm}

\pagestyle{fancy}
\fancyhf{}
\fancyhead[LE,RO]{Pedestrians Detector}
\fancyhead[RE,LO]{Team's name}
\fancyfoot[RE,LO]{MIRPR 2020-2021}
\fancyfoot[LE,RO]{\thepage}

\renewcommand{\headrulewidth}{2pt}
\renewcommand{\footrulewidth}{1pt}
\renewcommand{\headrule}{\hbox to\headwidth{%
  \color{lime}\leaders\hrule height \headrulewidth\hfill}}
\renewcommand{\footrule}{\hbox to\headwidth{%
  \color{lime}\leaders\hrule height \footrulewidth\hfill}}

\hypersetup{
pdftitle={artTitle},
pdfauthor={name},
pdfkeywords={pdf, latex, tex, ps2pdf, dvipdfm, pdflatex},
bookmarksnumbered,
pdfstartview={FitH},
urlcolor=cyan,
colorlinks=true,
linkcolor=red,
citecolor=green,
}
% \pagestyle{plain}

\setcounter{secnumdepth}{3}
\setcounter{tocdepth}{3}

\linespread{1}

% \pagestyle{myheadings}

\makeindex


\begin{document}

\begin{titlepage}
\sloppy

\begin{center}
BABE\c S BOLYAI UNIVERSITY, CLUJ NAPOCA, ROM\^ ANIA

FACULTY OF MATHEMATICS AND COMPUTER SCIENCE

\vspace{6cm}

\Huge \textbf{PEDESTRIANS DETECTOR}

\vspace{1cm}

\normalsize -- MIRPR report --

\end{center}


\vspace{5cm}

\begin{flushright}
\Large{\textbf{Team members}}\\
Name, specialisation, group, email
\end{flushright}

\vspace{4cm}

\begin{center}
2021-2022
\end{center}

\end{titlepage}

\pagenumbering{gobble}

\begin{abstract}
	Text of abstract. Short info about: 
	\begin{itemize}
		\item project relevance/importance, 
		\item inteligent methods used for solving, 
		\item data involved in the numerical experiments; 
		\item conclude by the the results obtained.
	\end{itemize}

	Please add a graphical abstract of your work. 
\end{abstract}


\tableofcontents

\newpage

\listoftables
\listoffigures
\listofalgorithms

\newpage

\setstretch{1.5}



\newpage

\pagenumbering{arabic}


 


\chapter{Introduction}
\label{chapter:introduction}

\section{What? Why? How?}
\label{section:what}


There are a lot of people out there dreaming about how the future will look like, and there is a very common answer to this question: self-driving cars. Currently, the automotive industry is trying to step out of the ordinary and offer an autonomous experience to the driver. A car is equipped with lots of sensors, just like humans - it can see things, or react to them. The idea is to teach the car to make decisions based on what these sensors intercept. This paper is trying to focus on one of the many branches of autonomous driving: endowing the car with the ability of seeing.
\begin{itemize}
	\item What is the (scientific) problem? 
	
	An autonomous car should be able to "see" and make its own decisions based on the input. This paper aims to provide a fast and reliable computer vision solution for pedestrians detection, which is one of the most crucial aspects when it comes to self-driving cars. Using the input from the camera, any pedestrian should be detected in less than the blink of an eye, and from a considerable distance, so that the moving car gets the possibility to react smoothly.
	\item Why is it important? 
	
	The goal is to get to a higher level of autonomy, but the hardest thing to do is keeping humans as safe as possible. The number of car crashes is huge every year, mostly because of lack of attention or driver's drowsiness. A machine will never get tired and this is why the automotive industry is trying to design the car to take over most of the driver's responsibilities. The thing here is that when it comes to human safety, the machine is not allowed to make mistakes, so the purpose is first of all to make these detection algorithms reach perfection.
	  
	\item What is your basic approach? 
	
	The idea is to create an intelligent algorithm that gets an image as input and outputs it with an emphasis on where the pedestrian has been detected. The algorithm should be able to provide really fast and accurate responses, therefore transfer learning techniques will also be used.  
\end{itemize}



\section{Paper structure and original contribution(s)}
\label{section:structure}

The research presented in this paper is focused on outlining the theory behind the TinyYoloV3 model and employing it for the particular problem of pedestrian detection in different contexts.

The main contribution of this report is to present a solution based on an intelligent classifier consisting of a pre-trained model which is run against multiple sets of data in the aim of solving the problem of pedestrian, vehicle and road sign detection.

The second contribution of this report is the development of a simple and intuitive mobile application that will present a practical user interface through which the user can easily test the algorithm results on input of their own.

The third contribution of this thesis consists of the employment of a number of optimizations with a view to increasing the overall accuracy of the algorithm and testing its performance in different scenarios.

The work is structured in seven chapters as follows: 

The first chapter is a short introduction in the subject of object detection in the driving assistance field, what it is about and why it is important and our reasons that were behind choosing this topic. 

The second chapter describes the scientific problem in more detai and considers the advantages and disadvantages of our aproachl.

The third chapter treats some other related work in the field and gives a brief description of their results.

In chapter four we provide the investigate approach, togheter with the tools and technologies that were used in order to implement it. We describe the underlying architecture of the TinyYoloV3 model and the algorithm employed by it, stating how we will use this for our problem and how it is suited for the driving assistance object of study. We show how the algorithm works in practice and provide a short list of the tools we will be using for our study.

Chapter five comprises the main part of this report and consists of the description of our application requirements, the methodology by which we plan to solve the problem, the datasets we will be conducting our experiments on and the results obtained in the end. At the end of the chapter, we also provide some discussion around the results and potential optimizations to the algorithm, comparing the results obtained with the initial ones. The chapter ends with a small presentation of the user interface.

Chapter 6 explains the experimental methodology and the numerical results obtained with our approach and the state of the art approaches. Our focus in this chapter is on the interpretation and the statistical validation of the results. Also, this chapter is a dive into the philosophical aspects of autonomous driving and how this is likely to affect the way in which we report ourselves to the task of driving in general. We analyze the objectivity of the solution proposed and raise some interesting questions relating to the ethics of the smart driving assistants in general. We also provide some interesting data about the way our algorithm performs on individuals of different races and ethnicities, by this trying to advance the idea of diversity and inclusion in the way we use such technology.

The last chapter offers a summarization of our conclusions and future work and also try to analyze the strenghts and weaknesses of our application with the focus on what we can improve bpth in the algorithm and the application.

\chapter{Scientific Problem}
\label{section:scientificProblem}


\section{Problem definition}
\label{section:problemDefinition}

The issue of road obstacle detection has been tended by many major car manufacturers, integrating them in modern car models in order to improve crash avoidance technology and increase confidence in their cars. In any case, most of most of these frameworks report destitute comes about in severely lit situations and are frequently flawed in their discovery processes.

The point of this paper is to show a implies of identifying people on foot in any kind of conditions and tie this to the current climate state to propose to the client the foremost suitable activities to be taken in certain activity circumstances or to respond naturally to them (further improvement).

The personal driving assistant would be built off an intelligent classification algorithm based on neural networks. Other methods used for obstacle detection include:
\begin{itemize}
	\item template matching approach - the real image is compared against a sufficient number of templates of the object-of-interest to identify the presence of the object in the sample image.
\end{itemize}

\textbf{\emph{Why an intelligent algorithm is more suitable for this?}} - The shape of the pedestrian, for example, is rarely predictable, it may be moving or standing, may differ in color or position, creating endless possibilities of representation. Thus, we need a more complex system of identifying such obstacles. The two main options are:

\begin{itemize}
	\item Feature Classifier approach - a classifier based on predefined features
	\item Deep Learning approach - a classifier with self learned features
\end{itemize}

We will be using the Deep Learning approach. Our classifier will receive as input the image or video sequence and output the same image, with the obstacles marked and delimited accordingly. This should provide a good start for further improvements and additional features that would contribute to a better navigation experience in autonomous driving.

\section{Challenges}

The challenges we have faced in putting together an algorithm that would help us achieve this goal are related to the specific use case of working with pre-trained models. The models themselves are pretty large in size and take up a great deal of resources to run. 

In trying to work with a model that is pre-trained, we have had little control over the time it takes the algorithm to do a detection. Much of effort was concentrated on trying to make the algorithm run faster and get a better accuracy with the amount of resources that we have had at our disposal. 

Lack of sufficient resources to run the algorithm was an imminent problem that we have come across, as well as the deployment part, where we have found it is difficult to deploy our app in the cloud due to lack of free options for dedicated GPU servers.

We have also placed a lot of focus on developing a "Diversity and inclusion" part, in which we try to compare the results the algorithm yields on people from different races compared to white people and how this could make our algorithm biased in terms of inclusion and multi-cultural coverage. This proved to be quite a challenging task, especially for arab women, where the algorithm would often fail to perform well at all.




\chapter{State of the art/Related work}
\label{chapter:stateOfArt}


The theory of the methods utilised until now in order to solve the given problem.

Answer the following questions for each piece of related work that addresses the same or a similar problem. 
\begin{itemize}
	\item What is their problem and method? 
	\item How is your problem and method different? 
	\item Why is your problem and method better?
\end{itemize}

In order to cite a given work you can use a bib file (see the example) and the $\ $ \textit{cite} command:
\cite{kennedy1}, \cite{Koh06}, \cite{Berlekamp82}, \cite{Storn95}, \cite{firefox}.



\chapter{Investigated approach}
\label{chapter:proposedApproach}

Describe your approach!

Describe in reasonable detail the algorithm you are using to address this problem. A psuedocode description of the algorithm you are using is frequently useful. Trace through a concrete example, showing how your algorithm processes this example. The example should be complex enough to illustrate all of the important aspects of the problem but simple enough to be easily understood. If possible, an intuitively meaningful example is better than one with meaningless symbols.


\chapter{Application (numerical validation)}
\label{chapter:application}


Explain the experimental methodology and the numerical results obtained with your approach and the state of art approache(s).

Try to perform a comparison of several approaches.

Statistical validation of the results.


\section{Methodology}
\label{section:methodology}

\begin{itemize}
	\item What are criteria you are using to evaluate your method? 
	\item What specific hypotheses does your experiment test? Describe the experimental methodology that you used. 
	\item What are the dependent and independent variables? 
	\item What is the training/test data that was used, and why is it realistic or interesting? Exactly what performance data did you collect and how are you presenting and analyzing it? Comparisons to competing methods that address the same problem are particularly useful.
\end{itemize}

\section{Data}
\label{section:data}

Describe the used data.

\section{Results}
\label{section:results}

Present the quantitative results of your experiments. Graphical data presentation such as graphs and histograms are frequently better than tables. What are the basic differences revealed in the data. Are they statistically significant?

\section{Discussion}
\label{section:discussion}

\begin{itemize}
	\item Is your hypothesis supported? 
	\item What conclusions do the results support about the strengths and weaknesses of your method compared to other methods? 
	\item How can the results be explained in terms of the underlying properties of the algorithm and/or the data. 
\end{itemize}



\chapter{Conclusion and future work}
\label{chapter:concl}

Try to emphasise the strengths and the weaknesses of your approach.
What are the major shortcomings of your current method? For each shortcoming, propose additions or enhancements that would help overcome it. 

Briefly summarize the important results and conclusions presented in the paper. 

\begin{itemize}
	\item What are the most important points illustrated by your work? 
	\item How will your results improve future research and applications in the area? 
\end{itemize}


\chapter{Latex examples}

Item example: 

\begin{itemize}
	\item content of item1
 	\item content of item2
 	\item content of item3
\end{itemize}



Figure example 

$\ldots$ (see Figure \ref{swarmsize})

\begin{figure}[htbp]
% 	\centerline{\includegraphics{Fig/FitEvol.eps}}  
	\caption{The evolution of the swarm size during the GA generations. This results were obtained for the $f_2$ test function with 5 dimensions.}
	\label{swarmsize}
\end{figure}


Table example: (see Table \ref{tab3PSO})


\begin{table}[htbp]
	\caption{The parameters of the PSO algorithm (the micro level algorithm) used to compute the fitness of a GA chromosome.}
	\label{tab3PSO}
		\begin{center}
			\begin{tabular}{p{220pt}c}

				\textbf{Parameter}& \textbf{Value} \\
				\hline\hline
 				Number of generations& 50 \\
 				Number of function evaluations/generation& 10 \\
 				Number of dimensions of the function to be optimized& 5 \\
 				Learning factor $c_{1}$& 2 \\
 				Learning factor $c_{2}$ & 1.8\\
 				Inertia weight& 0.5 + $\frac{rand()}{2}$\\
		
			\end{tabular}
		\end{center}
\end{table}

Algorithm example 

$\ldots$ (see Algorithm \ref{NGalg}).


\algsetup{indent=1em, linenosize=\footnotesize}

\begin{algorithm}
	\caption{SGA - Spin based Genetic AQlgorithm}
	\label{NGalg}
		\begin{algorithmic}


			\STATE \textbf{BEGIN}
  		\STATE @ Randomly create the initial GA population.
  		\STATE @ Compute the fitness of each individual.
  		\FOR{i=1 TO NoOfGenerations}
  			\FOR{j=1 TO PopulationSize}
  				\STATE p $\leftarrow$ RandomlySelectParticleFromGrid();
  				\STATE n $\leftarrow$ RandomlySelectParticleFromNeighbors(p);
  				\STATE @ Crossover(p, n, off);
  				\STATE @ Compute energy $\Delta H$
  				\IF {$\Delta H$ satisfy the Ising condition}
  					\STATE @ Replace(p,off);
  				\ENDIF
  			\ENDFOR
  		\ENDFOR
  		\STATE \textbf{END}
\end{algorithmic}
\end{algorithm}


\bibliographystyle{plain}
\bibliography{BibAll}

\end{document}
